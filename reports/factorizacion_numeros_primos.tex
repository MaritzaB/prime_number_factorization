\documentclass{article}
\usepackage[a4paper,top=1cm,bottom=2cm,left=3cm,right=3cm,marginparwidth=1.75cm]{geometry}  
\usepackage[spanish,es-tabla,es-nodecimaldot]{babel}
\usepackage{booktabs}
\usepackage{caption}
\usepackage{color} 
\usepackage{csvsimple}
\usepackage{float}
\usepackage{hyperref}
\usepackage{listings} 
\usepackage{multirow}
\usepackage{pgfplotstable}
\usepackage{setspace} 
%\usepackage{pythontex}

\definecolor{Code}{rgb}{0,0,0} 
\definecolor{Decorators}{rgb}{0.5,0.5,0.5} 
\definecolor{Numbers}{rgb}{0.5,0,0} 
\definecolor{MatchingBrackets}{rgb}{0.25,0.5,0.5} 
\definecolor{Keywords}{rgb}{0,0,1} 
\definecolor{self}{rgb}{0,0,0} 
\definecolor{Strings}{rgb}{0,0.63,0} 
\definecolor{Comments}{rgb}{0,0.63,1} 
\definecolor{Backquotes}{rgb}{0,0,0} 
\definecolor{Classname}{rgb}{0,0,0} 
\definecolor{FunctionName}{rgb}{0,0,0} 
\definecolor{Operators}{rgb}{0,0,0} 
\definecolor{Background}{rgb}{0.98,0.98,0.98} 


\lstset{frame=tb,
numbers=left, 
numberstyle=\footnotesize, 
numbersep=1em, 
xleftmargin=1em, 
framextopmargin=2em, 
framexbottommargin=2em, 
showspaces=false, 
showtabs=false, 
showstringspaces=false, 
frame=l, 
tabsize=4, 
% Basic 
basicstyle=\ttfamily\small\setstretch{1}, 
backgroundcolor=\color{Background}, 
% Comments 
commentstyle=\color{Comments}\slshape, 
% Strings 
stringstyle=\color{Strings}, 
morecomment=[s][\color{Strings}]{"""}{"""}, 
morecomment=[s][\color{Strings}]{'''}{'''}, 
% keywords 
morekeywords={import,from,class,def,for,while,if,is,in,elif,else,not,and,or,print,break,continue,return,True,False,None,access,as,,del,except,exec,finally,global,import,lambda,pass,print,raise,try,assert}, 
keywordstyle={\color{Keywords}\bfseries}, 
% additional keywords 
morekeywords={[2]@invariant,pylab,numpy,np,scipy}, 
keywordstyle={[2]\color{Decorators}\slshape}, 
emph={self}, 
emphstyle={\color{self}\slshape}, 
}




\begin{document}
\title{Factorización de números primos}
\author{Ana Maritza Bello Yañez}
\maketitle

\tableofcontents

\section{Descripción del problema}

La descomposición en factores primos es un concepto de la teoría de números que
se refiere a la descomposición de un número en el producto de dos números
primos. La factorización prima es un subconjunto de la factorización de enteros,
en la que un número compuesto se factoriza en el producto de dos enteros
cualesquiera. Parte del teorema que estipula que la descomposición en primos de
cualquier entero positivo mayor que 1, es única, excepto por el orden en que se
escriben los factores primos \cite{lewinter2015elementary}.

Los números primos juegan papeles importantes en algunas aplicaciones donde sus
propiedades especiales hacen que ciertas operaciones sean más fáciles o más
difíciles. Por ejemplo, algunos tipos de criptografía utilizan el producto de
dos grandes números primos para brindar seguridad. El hecho de que sea difícil
factorizar un número que es el producto de dos números primos grandes es lo que
hace que el algoritmo sea seguro \cite{stephens2019essential}.

Es un desafío encontrar los factores de los semiprimos (números que resultan del
producto de dos números primos) porque tienen solo un par de factores, y la
complejidad de encontrar los factores aumenta a medida que el tamaño del número
primo se usa en el producto aumenta. No existe un algoritmo de factorización
eficiente conocido para encontrar factores cuando los números son de cierto
tamaño. 

RSA (por sus siglas en inglés Rivest-Shamir-Adleman) es un criptosistema de
clave pública que se usa ampliamente para la transmisión segura de datos que usa
factorización prima, suponiendo que es realmente difícil encontrar la clave
privada del producto expuesto de números primos. Esta supuesta dificultad es la
razón detrás del uso de la factorización prima en criptografía
\cite{raj2019foundations}.


\section{Justificación del problema}
% ¿Por qué es importante resolver este problema?\

Los números primos en criptografía son vitales para la seguridad de nuestros
esquemas de cifrado. La factorización prima, también conocida como factorización
de enteros, es un problema matemático que se utiliza para proteger los esquemas
de cifrado de clave pública. Esto se logra mediante el uso de números semiprimos
extremadamente grandes que son el resultado de la multiplicación de dos números
primos \cite{sedgewick2015introduction}

RSA es una de las implementaciones iniciales de criptografía público-privada.
Utiliza el principio de factorización prima para generar un par de claves
pública-privada. El cifrado se realiza con la clave pública, que se distribuye a
todo el mundo, y el descifrado se realiza con la clave privada guardada en
secreto.

La idea de un criptosistema de clave pública-privada asimétrica se atribuye a
Whitfield Diffie y Martin Hellman, quienes publicaron este concepto en 1976.

El par de claves pública y privada se calculan con la ayuda de dos grandes
números primos. La clave pública se publica para el usuario y la clave privada
se mantiene en secreto. Los números primos también se mantienen en secreto.
Siempre que los números primos utilizados sean grandes, no es factible calcular
la clave privada a partir de la clave pública. Todo el criptosistema RSA se basa
en el problema de la teoría de números de la factorización de enteros, lo que
garantiza que la dificultad de la factorización prima sea proporcional al tamaño
de los números primos utilizados \cite{raj2019foundations}.


\section{Estado del arte}
% Qué se ha hecho hasta ahora?

Hasta la fecha no se ha publicado un algoritmo que pueda factorizar todos los
enteros en tiempo polinomial; es decir no hay algún algoritmo conocido que pueda
factorizar un número n gránde en tiempo $$ O(n^k)$$ para cualquier constante
\textit{k}.

Los algoritmos que resuelven este problema se dividen en dos dependiendo del
tamaño del número de entrada y las características de este.

Los algoritmos de categoría 1 o propósito específico son aquellos cuyo tiempo de
ejecución depende del tamaño del factor primo más pequeño. Por ejemplo:

\begin{itemize}
    \item División por tentativa
    \item Algoritmo Rho de Pollar
    \item Método de factorización de Fermat
\end{itemize}

Los algoritmos de categoría 2 o propósito general son aquellos cuyo tiempo de
ejecución depende solamente del tamaño del entero a ser factorizado. Por ejemplo:

\begin{itemize}
    \item Algoritmo de Dixon
    \item Factorización con fracciones contínuas
    \item Criba cuadrática
    \item Criba racional
\end{itemize}

\subsection{Dificultad y complejidad}


La forma más básica de factorizar un número, $N$, es intentar dividirlo por
todos los números inferiores a este hasta que encontremos un factor que divide
$N$. Luego intentamos dividir $N$ con el siguiente número y pronto terminaremos
con una lista de factores de $N$.

Para saber la complejidad de este algoritmo, recordemos que expresamos
complejidades en función de la longitud de la entrada. La longitud de bits del
número $N$ es $n= log2 N$. Por la definición básica de logaritmo, esto
significa que $N = 2^n$. Debido a que todos los números inferiores a $N/2$ son
suposiciones razonables para posibles factores de $N$, hay aproximadamente $$N/2
= 2^n/2$$ valores para probar. La complejidad de nuestro algoritmo de
factorización tentativo es, por lo tanto, $$ O(2n) $$ ignorando el coeficiente
1/2 en la notación $ O(2^n)$. \\

Por supuesto, muchos números son fáciles de tener en cuenta al encontrar primero
factores pequeños (2, 3, 5, etc.) y luego factorizando iterativamente cualquier
otro factor no primo. Pero aquí estamos interesados en el número de la forma $n
= p x q$, donde $p$ y $q$ son valores muy grandes, como se encuentran en la
criptografía.

Para la mayoría de los algoritmos, considera el tiempo de ejecución como una
función de la cantidad de entradas. Sin embargo, este algoritmo solo tiene una
entrada: el número que está factorizando. El algoritmo tarda más en factorizar
números grandes, por lo que no tiene sentido pensar en la cantidad de tiempo que
tarda el algoritmo en función del valor 1. 

Para algoritmos como este, el tiempo depende del tamaño de la entrada y no en el
número de entradas. Así es como los analiza: observando el tamaño de la entrada.
Podemos pensar en esto como mirar la cantidad de bits en el valor que va a
factorizar.

El peor caso para este algoritmo ocurre cuando el número de entrada es primo. En
ese caso, el programa examina posibles factores impares hasta llegar a la raíz
cuadrada del número. Esto significa que si el número de pasos que realiza para
llegar a ese punto es $ O(2^n)$, el tiempo de ejecución del algoritmo es
exponencial en el número de bits de entrada.

Su tiempo de ejecución depende de potencias de $\sqrt(n)$, por lo que no crece
tan rápido como lo haría si dependiera de potencias de 2 o algún número mayor.
Pero una vez que la función crece, realmente despega, como lo hacen todas las
funciones exponenciales. Es por eso que este algoritmo funciona para números
moderadamente grandes pero no es práctico para números muy grandes de 100
dígitos. Puede usar técnicas similares de usar bits de entrada para analizar
otros algoritmos que tienen un número fijo de entradas
\cite{stephens2015learning}.


\section{Solución}
% ¿Cómo se piensa resolver este problema?

\subsection{Algoritmos implementados}

Para este reporte implementé dos de los algorítmos mencionados en la sección
anterior:

\begin{itemize}
\item Factorización secuencial de números primos (ver
\texttt{sequentialFactorization()} de la sección \ref{sec:algoritmos})
    \item División por tentativa (ver
    \texttt{trialDivision()} de la sección \ref{sec:algoritmos})
\end{itemize}

\subsubsection{Algoritmo de factorización secuencial}

Este algoritmo consiste en dividir el número entre todos los números entre 2 y
n-1 menores a él. Cáda vez que un posible factor divide al número de manera
uniforme, se guarda el factor y se divide el número. Esto continua hasta
intentar la mayor cantidad de factores posibles \cite{stephens2019essential}.\\

Por ejemplo para encontrar los factores primos de 204, intentaría dividir 204
por 2, 3, 4, 5, y así sucesivamente, hasta llegar a 126. En este caso nos
detendríamos en el factor 3, como en este ejemplo:

\begin{center}
    \begin{tabular}{ c c c }
     Factores & Número \\ 
              & 204    \\
     2x2      & 102    \\
     2x2x3    & 51
    \end{tabular}
    \end{center}

\subsubsection{Algoritmo de división por tentativa}

Para mejorar el algoritmo de factorización secuencial, podemos hacer las
siguientes observaciones:

\begin{itemize}
    \item No necesitamos probar si el número es divisible por cualquier número par que
    no sea 2 porque, si es divisible por cualquier número par, también es divisible
    por 2. Esto significa que sólo necesitamos comprobar la divisibilidad por 2 y luego
    por números impares en lugar de por todos los factores posibles. Si lo hacemos,
    reducimos el tiempo de ejecución aproximadamente a la mitad.

    \item Solo necesita verificar los factores hasta la raíz cuadrada del número. Si
    $ n= p*q $, entonces \textit{p}  o \textit{q} deben ser menores o iguales
    que las $\sqrt(n)$. (Si \textit{p}  y \textit{q} son mayores que $\sqrt(n)$,
    entonces su producto es mayor que \textit{n}). Si marcamos los factores posibles
    hasta $\sqrt(n)$, encontraremos el factor más pequeño, y cuando dividamos
    \textit{n} por ese factor, encontraremos el otro. Esto reduce el tiempo de
    ejecución de $O(\sqrt(n))$.

    \item Cada vez que dividimos el número por un factor, podemos actualizar el
    límite superior de los posibles factores que necesitamos verificar, lo que nos
    lleva a un algoritmo mejorado de la factorización secuencial (ver
    \texttt{trialDivision()} de la sección \ref{sec:algoritmos}).

\end{itemize}

\subsection{Complejidad práctica del problema}

Con el fin de realizar el análisis de complejidad práctica, implementé la
función \texttt{timer()} y \texttt{timeScorer()} (ver en \ref{sec:complejidad}).

La función llamada \texttt{randBits()} (ver sección \ref{sec:complejidad}) la
implementé con el fin de obtener una lísta de números aleatorios generados con
base en los bits que utiliza en número. Es decir, con esta función generamos una
lista de \textit{n} elementos, cuyo elemento más pequeño es de una cifra y los
siguientes van aumentando hasta llegar a la cantidad deseada de cifras.\\

El flujo de trabajo para el análisis del problema sigue los siguientes pasos:

\begin{enumerate}

\item Ejecutar función \texttt{randBits()} cuyo objetivo es generar una lista de
números aleatorios con elementos que incrementan en la cantidad de cifras del
número, con el fin de aumentar la complejidad.
\item Ejecutar función \texttt{timer()} cuyo objetivo es tomar el tiempo de
ejecución de cada uno de los algoritmos de factorización implementados para este
trabajo para un número dado.
\item Ejecutar función \texttt{timeScorer()} que toma el tiempo de ejecución
para los diferentes tamaños de numeros a factorizar.
\end{enumerate}
    

Los resultados obtenidos del análisis de la complejidad práctica se muestran en
la sección \ref{sec:resultados}


\section{Resultados y conclusiones} \label{sec:resultados}

Con el fin de medir la complejidad de cada uno de los algoritmos, evaluamos cada
uno con una lista de números enteros aleatorios, cuyo tamaño de almacenamiento
va desde uno hasta 60 bits.

Medimos el tiempo que tardaba cada uno de los algoritmos en encontrar los
factores primos del número evaluado.

En la figura podemos observar la complejidad práctica del algoritmo de
factorización secuencial y del algoritmo de división tentativa.

\begin{figure}[H]
        \includegraphics[scale=1]{figures/practical_complexity.png}
\caption{Complejidad práctica de los algorimoes de feactorizacioń secuencial y
de división tentativa. En el eje x está representado el tamaño del número en
bits que fue evaluado por el algoritmo. En el eje y está representado el tiempo
que tardó el algoritmo en encontrar los factores del número evaluado.
Los ejes se encuentran en escala logarítmica para poder apreciar con mayor
detalle la proporción por la que un algoritmo es más rápido que otro.
    } \label{complejidad}
\end{figure}

\subsection{Conclusiones}

De la figura \ref{complejidad} podemos concluir que:

\begin{enumerate}
    \item Los dos algoritmos tienen la misma tendencia en la complejidad.
\item El algoritmo de división tentativa tiende a ser más rápido que el de
factorización secuencial conforme el tamaño del número aumenta.
\item La evaluación con los números cuyo tamaño van de 1 a 10 bits en algunos
casos, el algoritmo de factorización resulta ser más rápido que el de división
tentativa.
\item Se comprueba que la complejidad del problema aumenta conforme aumenta el
tamaño del número.
\item En la práctica, fue muy dificil evaluar los algoritmos con valores mayores
a los 55 bites. Esto puede ser debido a la arquitectura de la computadora
utilizada para la evaluación.
\item Como trabajo a futuro, sería interesante poder comparar otros algoritmos de
factorización de números enteros. Ya que aunque pudimos observar que en efecto
un algoritmo es más rápido que el otro, el principio en ambos es el mismo.

\end{enumerate}

\pagebreak
\bibliography{../references/references.bib} 
\bibliographystyle{apalike}

\pagebreak
\section{Apéndice}

El trabajo presentado en este reporte también se puede consultar y clonar en
este el repositorio de GitHub de \url{https://github.com/MaritzaB/prime_number_factorization/tree/feature/init_project}

\subsection{Algoritmos de factorización implementados} \label{sec:algoritmos}

\lstinputlisting[language=Python]{../src/prime_number_fact.py}

\pagebreak

\subsection{Implementación del código de análisis de complejidad práctica} \label{sec:complejidad}

\lstinputlisting[language=Python]{../src/analizer.py}

\pagebreak
\subsection{Código para graficar la complejidad práctica} \label{sec:grafica}

\lstinputlisting[language=Python]{../src/plotter.py}


\end{document}